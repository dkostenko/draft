\documentclass[12pt,a4paper,oneside,titlepage]{article}
\usepackage[utf8]{inputenc}
\usepackage[english]{babel}
\usepackage{amsmath}
\usepackage{amsfonts}
\usepackage{amssymb}
\usepackage{textcase} 
\usepackage{tocloft}
\usepackage[left=3.5cm,right=1cm,top=2cm,bottom=1.2cm]{geometry}
\author{Kostenko}
\setcounter{tocdepth}{3}%для глубины страницы содаржения
\renewcommand{\baselinestretch}{1.25}
\renewcommand{\cftsecleader}{\cftdotfill{\cftdotsep}} % точки в содержании для секций
\begin{document}
{
\thispagestyle{empty}
\newpage
\centering

\textbf{
National Research University Higher School of Economics\\
}
Faculty of Business Informatics\\
School of Software Engineering\\
Software Management Department

\vfill


\begin{large}
\MakeTextUppercase{
Based on Lucas-Kanade algorithm Object dynamic identification application
}
\end{large}


\vfill

\begin{tabular}{lr}
Student: & Kostenko Dmitry \\
Group: & 472SE \\
Argument Consultant: & Prof. Ivan. M. Gostev, PhD \\
Style and Language Consultant: & Tatiana A. Stepantsova
\end{tabular}

\vspace{\fill}

Moscow\\ \number\year
\clearpage
}


\section*{Abstract}
{
1 paragraph
asdasdasd
%1) мотивация/проблема
% 
%2) методы/процедуры/подход
% Для отслеживания объекта будет использоват подход Лукаса и Канаде.
%3) результаты/продукт
% Результатом работы будет являться программа, которая отслеживает количество пальцев на ладони, которое показывает человек в камеру.
%4) последствия (писать про то, что код будет в открытом доступе)
%5) практический последствия ?????
% окд программы будет лежать в открытом доступе под лицензией GNU. т.к. не существует открытых исходных кодов программы, которая определяет количество несогнутых пальцев на ладони, то мой код может стать примером реализации подхода Лукаса и Канаде.
%6) чего нового, в том, что я делаю?
}


{
\newpage
\centering
\tableofcontents
}


\newpage
\section*{Introduction}
\addcontentsline{toc}{section}{Introduction}
1-2 pages
adssdadas


\newpage
\section*{Main body}
\addcontentsline{toc}{section}{Main body}
5 pages

\newpage
\section*{Conclusion}
\addcontentsline{toc}{section}{Conclusion}
1 pages
%


\newpage
\section*{Bibliography}
\addcontentsline{toc}{section}{Bibliography}
1-2 pages



\end{document}
