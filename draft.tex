\documentclass[12pt,a4paper,oneside,titlepage]{article}
\usepackage[utf8]{inputenc}
\usepackage[english, russian]{babel}
\usepackage{amsmath}
\usepackage{amsfonts}
\usepackage{amssymb}
\usepackage{textcase} 
\usepackage{tocloft}
\usepackage{lastpage}
\usepackage[left=3.5cm,right=1cm,top=2cm,bottom=2cm]{geometry}
\author{Kostenko}
\setcounter{tocdepth}{3}%для глубины страницы содаржения
\renewcommand{\baselinestretch}{1.25}
\renewcommand{\cftsecleader}{\cftdotfill{\cftdotsep}} % точки в содержании для секций
\bibliographystyle{unsrt}
\begin{document}
{
\thispagestyle{empty}
\newpage
\centering

\textbf{
National Research University Higher School of Economics\\
}
Faculty of Business Informatics\\
School of Software Engineering\\
Software Management Department

\vfill


\begin{large}
\MakeTextUppercase{
An Application for Dynamic Object Identification Based on Lucas-Kanade Algorithm
}
\end{large}


\vfill

\begin{tabular}{lr}
Student: & Kostenko Dmitry \\
Group: & 472SE \\
Argument Consultant: & Prof. Ivan. M. Gostev, PhD \\
Style and Language Consultant: & Tatiana A. Stepantsova
\end{tabular}

\vspace{\fill}

Moscow\\ \number\year
\clearpage
}

\section*{Abstract}
{
В данной статье описывается подход к обнаружению и подсчету транспортных средств на атодорогах.
Он основан на дифференциальном методе вычисления оптического потока, предложенном Лукасом и Канаде.
Отличие данного метода от других состоит в том, что нет необходимости подготавливать модель фона.
}


{
\newpage
\centering
\tableofcontents
}


\newpage
\section*{Introduction}
\addcontentsline{toc}{section}{Introduction}
В наше время наблюдается высокие рост количества транспортных средств во всех городах России.
По данным ГИБДД только в Москве ежегодный прирост автомобилей составляет 110 - 120 тысяч.
В результате проблема заторов автотранспортных дорог становится более острой.
Как следствие увеличивается расход топлива, уровень загрязнения окружающей среды и время пути каждого автомобилиста.
Одним из решений данной проблемы является установка городской интелектуальной транспортной системы (ИТС).
ИТС варьируются от простых систем регулирования светофоров, до систем регистрации скорости потоков транспортных средств, контроля автомобильного потока и распознования фактов нарушений.
%(http://ru.wikipedia.org/wiki/Интеллектуальная_транспортная_система)
Такие системы позволяют решать следующие задачи.

1) Обеспечение максимальной пропускной способности.

1) Безопасность. Основная цель — снижение аварийности на дорогах. Сюда же входит мониторинг природных катаклихмов и человеческого фактора.

2) Мобильность. Сбор информации о пробках от движущихся в потоке автомобилей и информирование участников движения.

3) Защита окружающей среды. Снижение ущерба окружающей среде от автотранспорта посредством мониторинга ситуации в реальном времени и своевременного принятия решений.

ИТС может содержать в себе датчики различных типов, от тепловых до ультразвуковых.
Ручная обработка гигантского объема данных, поступающих от всех сенсоров таких систем непрактична.
Поэтому появляется необходимость автоматизировать обработку данных и заключения выводов на основе них.

Автоматическое обнаружение транспортных средств в данных видеонаблюдения является комплексной задача в компьютерном зрении.

Одной из задач такой системы является подсчет транспортных средств на автодороге.
Которая в свою очередь тоже разбивается на подзадачи компьютерного зрания, такие как: выделение объектов переднего плана (автомобилей) и отслеживание их положения в последующих кадрах.


В данной статье мы описываем подход к решению задачи автоматического отслеживания движущихся транспортных средств и их подсчета.
Где единственным источником данных о ситуации на автодороге является видеокамера. 
Краткое содержание последующих глав:

3) Обзор существующих решений в области видеонаблюдения на автодорогах.

4) Постановка задачи и формулирование требований к разрабатываемому алгоритму.

5) Изложение методов и алгоритмов решения поставленной задачи

6) Краткое содержание планируемых рещультатов работы и выводы.

\newpage
\section*{Problem statement}
\addcontentsline{toc}{section}{Problem statement}














Одна из наиболее важных задач в видео наблюдении состоит из идентификации объекта интереса и отслеживание его траектории в последующих кадрах.
Эту задача может быть разбита на подзадачи.
Первое, обнаружение объектов интереса.
После того, как мы это сделали мы получим координаты объектов на изображении.

О том, как выделять объект интереса (ограничевающей рамкой или попиксельной маской) будет написанно позже.

Затем следить за изменениями координат обектов интереса в последующих кадрах.

Для формализации приведем некоторые определения.

Т.к. видео - это последовательность кадров, то в каждый момент времени мы имеем один кадр, называемый текущим.
Если кадр i - текущий кадр, то кадр (i-1) - предыдущий.

Объектами переденго плана будем считать любые движущиеся объекты (транспортные средства, деревья, люди).
Определение объекта, как движущегося, зависит от характеристик камеры и окружающей обстановки.
Поэтому движущимися объектами назовем те объекты, которые меняют свое положение на текущем кадре, относительно предыдущего кадра.

Движущийся объект перед камерой или движущиеся камера в неподвижной обстановке ведет к изменения изображения.
Изображение видимого движения объекта называется оптическим потоком.

Существует несколько методов вычисления оптического потока.
Лукас и Канаде предложили дифференциальный подход.
О нем будет рассказано дальше.



Для вычисления оптического потока необходимо сделать несколько предположений.

1) изображение - это непрерывная функция от двух переменных;
2) яркость объекта остается неизменной в небольшой промежуток времени;
3) отслеживаеммый объект на новом кадре будет расположен на небольшом расстоянии относительно предыдущего кадра.

Первое предположение дает нам возможность использовать методы математического анализа и позволяет производить матемстические операции над изображением.
Второе предположение существует потому, что мы живем в реальном мире, в котором  обхекты не могут мгновенно перемещаться на большие расстояния.
Третье предположение необходимо потому что мы не сможем без него отслеживать объект.










Сформулируем требования к алгоритму:
1) Работать без каких-либо предварительных данных об автодороге
2) Обработка потока данных в реальном времени
3) Не должен требовать высоких вычислительных мощностей. Минимальные требования к оборужованию будут предложены в техническом задании.


\newpage
\section*{Related work}
\addcontentsline{toc}{section}{Related work}
В данной главе представлен краткий обзор существующих решений в области видеонаблюдения на автодорогах.

В мире существует только одна всеобъемлющая архитектура ИТС. Предложенная транспортным департаментом США инициатива, направленная на создание единого информационного пространства, объединяющего автомобили, дорожное оборудование, диспетчерские залы и центры обработки данных по всей стране. 
%(http://www.iteris.com/itsarch/documents/physical/physical.pdf)



\newpage
\section*{Algorithm}
\addcontentsline{toc}{section}{Algorithm}
В данной главе раскрывается подход к обнаружению и подсчету транспортных средств на атодорогах.



Т.к. видео - это последовательность кадров, то в каждый момент времени мы имеем одно изображение.
При получении нового кадра, сперва необходимо избавиться от шумов.
Для этого сгладим изображение фильтром Гаусса. %ссылка на гауса

Так как в дальнейшем мы не будем использовать информацию о цвете объектов, то для уменьшения количества избыточной информации переведем изображение из цветовой модели RGB в градации серого. %сказать каким алгоритмом

ПОСЛЕ УМЕНЬШИМ ИЗОБРАЖЕНИЕ

\newpage
\section*{Allocation of foreground objects}
\addcontentsline{toc}{section}{Allocation of foreground objects}
Целью метода, описанного в данном блоке, является получение положений объектов, относящихся к переднему плану.

Традиционные алгоритмы получения объектов переднего плана основаны на методе попиксельного вычитания изображений.
Но у данного метода есть недостаток.
С помощью него мы можем получить большое количество несвязных областей. %показать картинку

Поэтому, для получения более связных областей, необходимо модифицировать данный метод.
Сперва разобьем каждый кадр на непересекающиеся блоки.
Оптимальный размер блока определяется империческии и зависит от характеристи камеры и окружающей обстановки.
Поэтому, для начала, зададим размер блока равный 3х3 пикселей.
Затем попиксельно вычтем предыдущий кадр из текущего.

В конце, отфильтруем пиксели каждого блока у полученной разности изображений таким образом:
если количество пикселей в каждом блоке, относящихся к объектам переднего плана, превышает заданный порог, то будем считать, что блок принадлежит переднему плану.
В инном случаем будем считать обратное.

В результате мы получим бинарное изображение объектов переднего плана.%показать картинку
Но на данном изображении могут присутствовать шумы.
Поэтому необходимо обработать изображение морфологическими операциями.
Применим эррозию, а затем наращивание. %delation





Так же необходимо заметить, что метод вычитания предыдущего кадра из текущего дает хороший отклик на границе движущегося объекта.
Но в оласти движущегося объхекта наблюдается обратный эффект. %показать картинку
В случае, когда транспортное средство имеет большие размеры, вероятность определить транспортное средство, как объект переднего плана, существенно уменьшается.

Чтобы избежать подобного эффекта введем еще 2 понятия: краткосрочная модель переднего плана и долгострочная модель  переднего плана.

Краткосрочной моделью переднего плана является такая модель, которая получается в результате применения последовательности действий, описанных выше.
Долгосрочной моделью переднего плана назовем попиксельную сумму N моедлей переднего плана, где N - натуральное число и больше 1.
N зависит от характеристик камеры и окружающей среды вычисляется имерическим путем.

Т.к. отслеживаемые объекты движутся, то при вычислении долгосрочной модели переднего плана мы получим более свзные области предполагаемых транспортных средств. % показать кратинку
На изображении видно, что область объекта может разделиться на несколько областей.
Для уменьшения количества связных областей применим морфологическую операцию наращивание.




\newpage
\section*{Foreground segmentation}
\addcontentsline{toc}{section}{Foreground segmentation}
В данной главе излагается метод сегментации объектов переднего плана.












Приведем содержание алгоритма, затем опишем каждый пункт.

1) Перевести изображение в серое
2) обработать изображение фильтром гауса (против шумов)
3) вычесть из текущего изображения предыдущее
4) наложить друг на друга n предыдущих разностей (для получения транспортного средства) Делать это каждые 5 кадров
5) выделить связные области, которые и будут предположительно транспортными средствами 
6) сопоставить текущия связные области с такими же на предыыдущих кадрах или инициализировать новые
7) если такая связна область пересекает линию интереса, то прибавляем 1 к счетчику



\newpage
\section*{Conclusion}
\addcontentsline{toc}{section}{Conclusion}
1 page


\newpage
\renewcommand\refname{Bibliography}
\bibliographystyle{plain}
\bibliography{draft}



\end{document}
