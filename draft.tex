\documentclass[12pt,a4paper,oneside,titlepage]{article}
\usepackage[utf8]{inputenc}
\usepackage[english, russian]{babel}
\usepackage{amsmath}
\usepackage{amsfonts}
\usepackage{amssymb}
\usepackage{textcase} 
\usepackage{tocloft}
\usepackage{lastpage}
\usepackage[left=3.5cm,right=1cm,top=2cm,bottom=2cm]{geometry}
\author{Kostenko}
\setcounter{tocdepth}{3}%для глубины страницы содаржения
\renewcommand{\baselinestretch}{1.25}
\renewcommand{\cftsecleader}{\cftdotfill{\cftdotsep}} % точки в содержании для секций
\bibliographystyle{unsrt}
\begin{document}
{
\thispagestyle{empty}
\newpage
\centering

\textbf{
National Research University Higher School of Economics\\
}
Faculty of Business Informatics\\
School of Software Engineering\\
Software Management Department

\vfill


\begin{large}
\MakeTextUppercase{
An Application for Dynamic Object Identification Based on Lucas-Kanade Algorithm
}
\end{large}


\vfill

\begin{tabular}{lr}
Student: & Kostenko Dmitry \\
Group: & 472SE \\
Argument Consultant: & Prof. Ivan. M. Gostev, PhD \\
Style and Language Consultant: & Tatiana A. Stepantsova
\end{tabular}

\vspace{\fill}

Moscow\\ \number\year
\clearpage
}

\section*{Abstract}
{
В данной статье описывается подход к обнаружению и подсчету транспортных средств на атодорогах.
Он основан на дифференциальном методе вычисления оптического потока, предложенном Лукасом и Канаде.
Отличие данного метода от других состоит в том, что нет необходимости подготавливать модель фона.
}


{
\newpage
\centering
\tableofcontents
}


\newpage
\section*{Introduction}
\addcontentsline{toc}{section}{Introduction}
В наше время наблюдается высокие рост количества транспортных средств во всех городах России.
По данным ГИБДД только в Москве ежегодный прирост автомобилей составляет 110 - 120 тысяч.
В результате проблема заторов автотранспортных дорог становится более острой.
Как следствие увеличивается расход топлива, уровень загрязнения окружающей среды и время пути каждого автомобилиста.
Одним из решений данной проблемы является установка городской интелектуальной транспортной системы (ИТС).
ИТС варьируются от простых систем регулирования светофоров, до систем регистрации скорости транспортных средств, контроля автомобильного потока и распознования фактов нарушений.
%(http://ru.wikipedia.org/wiki/Интеллектуальная_транспортная_система)
Такие системы с полной комплектацией позволяют контролировать три важных направления:

1) Безопасность. Основная цель — снижение аварийности на дорогах. Сюда же входит мониторинг природных и техногенных катаклизмов.

2) Мобильность. Сбор информации о пробках от движущихся в потоке автомобилей и информирование участников движения.

3) Защита окружающей среды. Снижение ущерба окружающей среде от автотранспорта посредством мониторинга ситуации в реальном времени и своевременного принятия решений.

ИТС может содержать в себе датчики различных типов, от тепловых до ультразвуковых.
Ручная обработка гигантского объема данных, поступающих от всех сенсоров таких систем непрактична.
Поэтому появляется необходимость автоматизировать обработку данных и заключения выводов на основе них.

Автоматическое обнаружение транспортных средств в данных видеонаблюдения является комплексной задача в компьютерном зрении.

Одной из задач такой системы является подсчет транспортных средств на автодороге.
Которая в свою очередь тоже разбивается на подзадачи компьютерного зрания, такие как: выделение объектов переднего плана (автомобилей) и отслеживание их положения в последующих кадрах.


В данной статье мы описываем подход к решению задачи автоматического отслеживания движущихся транспортных средств и их подсчета.
Где единственным источником данных о ситуации на автодороге является видеокамера. 
Краткое содержание последующих глав:

3) Краткое описание существующих решений в области видеонаблюдения на автодорогах.

4) Постановка задачи и формулирование требований к разрабатываемому алгоритму.

5) 

6) Краткое содержание важных моментов статьи, перспективы развития подхода.

\newpage
\section*{Problem statement}
\addcontentsline{toc}{section}{Problem statement}
Требования к алгоритму:
1) Работать без каких-либо предварительных данных об автодороге
2) Обработка потока данных в реальном времени
3) Не должен требовать высоких вычислительных мощностей


\newpage
\section*{Related work}
\addcontentsline{toc}{section}{Related work}
В данной главе представлен краткий обзор существующих решений в области видеонаблюдения на автодорогах.

В мире существует только одна всеобъемлющая архитектура ИТС. Это предложенная транспортным департаментом США инициатива, направленная на создание единого информационного пространства, объединяющего автомобили, дорожное оборудование, диспетчерские залы и центры обработки данных по всей стране. 
%(http://www.iteris.com/itsarch/documents/physical/physical.pdf)



\newpage
\section*{Algorithm}
\addcontentsline{toc}{section}{Algorithm}
В данной главе более подробно раскрывается подход к обнаружению и подсчету транспортных средств на атодорогах.

Традиционные 




\newpage
\section*{Conclusion}
\addcontentsline{toc}{section}{Conclusion}
1 page


\newpage
\renewcommand\refname{Bibliography}
\bibliographystyle{plain}
\bibliography{draft}



\end{document}
