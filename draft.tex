\documentclass[12pt,a4paper,oneside,titlepage]{article}
\usepackage[utf8]{inputenc}
\usepackage[english]{babel}
\usepackage{amsmath}
\usepackage{amsfonts}
\usepackage{amssymb}
\usepackage{textcase} 
\usepackage{tocloft}
\usepackage{lastpage}
\usepackage[left=3.5cm,right=1cm,top=2cm,bottom=2cm]{geometry}
\author{Kostenko}
\setcounter{tocdepth}{3}%для глубины страницы содаржения
\renewcommand{\baselinestretch}{1.25}
\renewcommand{\cftsecleader}{\cftdotfill{\cftdotsep}} % точки в содержании для секций
\bibliographystyle{unsrt}
\begin{document}
{
\thispagestyle{empty}
\newpage
\centering

\textbf{
National Research University Higher School of Economics\\
}
Faculty of Business Informatics\\
School of Software Engineering\\
Software Management Department

\vfill


\begin{large}
\MakeTextUppercase{
Based on Lucas-Kanade algorithm Object dynamic identification application
}
\end{large}


\vfill

\begin{tabular}{lr}
Student: & Kostenko Dmitry \\
Group: & 472SE \\
Argument Consultant: & Prof. Ivan. M. Gostev, PhD \\
Style and Language Consultant: & Tatiana A. Stepantsova
\end{tabular}

\vspace{\fill}

Moscow\\ \number\year
\clearpage
}

\section*{Abstract}
{
In this paper we present a novel approach of object identification and tracking.
We use a differential method for optical flow calculation developed by B. D. Lucas and T. Kanade \cite{lucasKanade}.
The proposed approach consists of three steps which can be executed in real-time.
As a first step the 2D vector of optical flow is calculated by Lucas-Kanade method \cite{lucasKanade}.
Then it produces a binary vector of the optical flow vector to generate regions that are moving.  
Finally, program divides moving regions of objects, using differences in speed.
To demonstrate the moving objects, they can be marked with different colors.
}


{
\newpage
\centering
\tableofcontents
}


\newpage
\section*{Introduction}
\addcontentsline{toc}{section}{Introduction}
In our generation scientific and engineering community attempts to build human-like machines.
Like a man, these robots are supposed to have various senses, ability to analyse incoming information, based on its conclusions and develop and implement behaviour patterns.
And starting to implement, scientists immediately came across a problem of image understanding.
There are a lot of engineering achievements that supersede analogues of the most perfect system - human.
However, there is a large gap between human and machine in the technology of artificial intelligence.
Impossibility to fully automate analysis of information from visual channel, even as a child level, pushes researchers to move forward gradually.
They divide the problem into sub-problems of computer vision. 
For example, improving image quality, identifying features points and determining the similarity of images.

The vision channel is one of the most informative ones.
Data volume from video stream exceeds by several times volumes from other sensors.
Here lies the pitfall - redundancy of information.
Occasionally just a few bytes of information are enough.
For example, we need only the object of interest, and in addition to this, we have other objects, background and fine detail.
In contrast to human, the machine can not optimally handle this task.
Consequence of this problem is the fact that even an ant can navigate better in the complicated situation than existing robots.

Why is the task of understanding the image so complex?
In the image processing we can come across several problems.
Firstly, objects of observations are very volatile in general.
Causes of variability are difficult to formalize.
Existing methods of dealing with different lighting, noises and distortions cope rather clumsily.
Secondly, scene observations also can not be modelled.
This is due to the fact that there is a diversity of geometrical forms, colors and textures.
Third problem is knowledge base.
Any system required to recognize an object should be aware of it.
It means that system should be trained to recognize images.
People, during their life, spend a lot of time doing it.

Nowadays, paradigm proposed by D. Marr \cite{marr} is used in image processing.
Its point is to process images in sequence:
\begin{enumerate}
  \item image pre-processing
  \item segmentation
  \item distinguishing of the geometric structure
  \item determining the semantics
\end{enumerate}

In other words, processing of low-level should be performed at first, then intermediate-level processing and finally high-level processing.

In the general form, motion analysis is a comparison of consecutive images of the observed scene, which is the difference between the current frame and the previous frame.
As the result of this operation, we have contours of a moving object.
This is one of the basic ideas of our work.

The rest of this paper is organised as follows:
\begin{description}
  \item[Section 2] \hfill \\
  Identification of key problems of object tracking and problem statement.
  \item[Section 3] \hfill \\
  An algorithm for solving the problems and its implementation. 
  \item[Section 4] \hfill \\
  Summing-up, prospects of our algorithm and formulating the scope of our method.
\end{description}

\newpage
\section*{Problem statement}
\addcontentsline{toc}{section}{Problem statement}
2-3 pages



\newpage
\section*{Algorithm}
\addcontentsline{toc}{section}{Algorithm}
2-3 pages



\newpage
\section*{Conclusion}
\addcontentsline{toc}{section}{Conclusion}
1 page


\newpage
\renewcommand\refname{Bibliography}
\bibliographystyle{plain}
\bibliography{draft}



\end{document}
