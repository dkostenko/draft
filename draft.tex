\documentclass[12pt,a4paper,oneside,titlepage]{article}
\usepackage[utf8]{inputenc}
\usepackage[english]{babel}
\usepackage{amsmath}
\usepackage{amsfonts}
\usepackage{amssymb}
\usepackage{textcase} 
\usepackage{tocloft}
\usepackage{lastpage}
\usepackage[left=3.5cm,right=1cm,top=2cm,bottom=1.2cm]{geometry}
\author{Kostenko}
\setcounter{tocdepth}{3}%для глубины страницы содаржения
\renewcommand{\baselinestretch}{1.25}
\renewcommand{\cftsecleader}{\cftdotfill{\cftdotsep}} % точки в содержании для секций
\bibliographystyle{unsrt}
\begin{document}
{
\thispagestyle{empty}
\newpage
\centering

\textbf{
National Research University Higher School of Economics\\
}
Faculty of Business Informatics\\
School of Software Engineering\\
Software Management Department

\vfill


\begin{large}
\MakeTextUppercase{
Based on Lucas-Kanade algorithm Object dynamic identification application
}
\end{large}


\vfill

\begin{tabular}{lr}
Student: & Kostenko Dmitry \\
Group: & 472SE \\
Argument Consultant: & Prof. Ivan. M. Gostev, PhD \\
Style and Language Consultant: & Tatiana A. Stepantsova
\end{tabular}

\vspace{\fill}

Moscow\\ \number\year
\clearpage
}

\section*{Abstract}
{
In this paper we present a novel approach of object identification and tracking.
We use a differential method for optical flow calculation developed by B. D. Lucas and T. Kanade \cite{lucasKanade}.
The proposed approach consists of three steps which can be executed in real-time.
As a first step the 2D vector of optical flow is calculated by Lucas-Kanade method \cite{lucasKanade}.
Then it produces a binary vector of the optical flow vector to generate regions that are moving.  
Finally, program divides moving regions of objects, using differences in speed.
To demonstrate the moving objects, they can be marked with different colors.
}


{
\newpage
\centering
\tableofcontents
}


\newpage
\section*{Introduction}
\addcontentsline{toc}{section}{Introduction}
In our generation scientific and engineering community attempts to build a human-like machine.
Like a man, these robots are supposed to have various senses, ability to analyse incoming information, based on its done conclusions and develop and implement behaviour patterns.
And starting to realize, scientists immediately came across a problem of image understanding.
There are a lot of engineering achievements that supersede analogues in the most perfect system - human.
However, there is a large gap in the technology of artificial intelligence.
Impossibility to fully automatic analysis of information from visual channel, even as a child does, pushes researchers to move forward gradually.
They divide the problem into sub-problems of computer vision. 
For example, improve image quality, identify features points and determine the similarity of images.

The vision channel is one of the most informative one.
Data volume from video stream exceeds by several times volumes from other sensors.
Here lies the pitfall - redundancy of information.
%Иногда бывает достаточно нескольких байт информации
Occasionally it's just a few bytes of information.
For example, we need only the object of interest, and in addition to this, we have other objects, background and fine details.

\newpage
\section*{Main body}
\addcontentsline{toc}{section}{Main body}
5 pages

\newpage
\section*{Conclusion}
\addcontentsline{toc}{section}{Conclusion}
1 pages
%


\newpage
\renewcommand\refname{Bibliography}
\bibliographystyle{plain}
\bibliography{draft}



\end{document}
