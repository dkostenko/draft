\documentclass[12pt,a4paper,oneside,titlepage]{article}
\usepackage[utf8]{inputenc}
\usepackage[english]{babel}
\usepackage{amsmath}
\usepackage{amsfonts}
\usepackage{amssymb}
\usepackage{textcase} 
\usepackage{tocloft}
\usepackage{lastpage}
\usepackage[left=3.5cm,right=1cm,top=2cm,bottom=1.2cm]{geometry}
\author{Kostenko}
\setcounter{tocdepth}{3}%для глубины страницы содаржения
\renewcommand{\baselinestretch}{1.25}
\renewcommand{\cftsecleader}{\cftdotfill{\cftdotsep}} % точки в содержании для секций
\bibliographystyle{unsrt}
\begin{document}
{
\thispagestyle{empty}
\newpage
\centering

\textbf{
National Research University Higher School of Economics\\
}
Faculty of Business Informatics\\
School of Software Engineering\\
Software Management Department

\vfill


\begin{large}
\MakeTextUppercase{
Based on Lucas-Kanade algorithm Object dynamic identification application
}
\end{large}


\vfill

\begin{tabular}{lr}
Student: & Kostenko Dmitry \\
Group: & 472SE \\
Argument Consultant: & Prof. Ivan. M. Gostev, PhD \\
Style and Language Consultant: & Tatiana A. Stepantsova
\end{tabular}

\vspace{\fill}

Moscow\\ \number\year
\clearpage
}


\section*{Abstract}
{
%Как научная дисциплина компьютерное зрение является динамично развивающейся.
Computer vision is a dynamic scientific discipline.
%Одной из главных задач компьютерного зрения является слежение за объектом.
One of the most important objectives of the computer vision is real-time object tracking.
%Основой разрабатываемого алгоритма будет метод, предложенный Лукасом и Канаде.
A basis of the developed algorithm is the differential local method of Lucas and Kanade \cite{lucasKanade}.
%Данный метод позволяет отслеживать точечные особенности в последовательности кадров.
This method allows to keep track of point features in the frame sequence. 
The aim of this degree work is to develop a program that implements this functionality:
able to read a video-stream from a laptop's built-in camera,
in the presence of a human hand on the frame, program will display the number of unbent fingers.
%Но т.к. необходимо следить не за одной точечной особенностью, а за определенной группой точечных особенностей, то необходимо создать математическую модель кисти руки человека.
%И в дальнейшем отслеживать полученную математическую модель.
%Результатом будет программа, имеющая следующий функционал. 
%Возможность считывать видео-поток со встроенной камеры ноутбука.
%При присутствие на кадре человеческой ладони программу будет выводить количество несогнутых пальцев.
}


{
\newpage
\centering
\tableofcontents
}


\newpage
\section*{Introduction}
\addcontentsline{toc}{section}{Introduction}
1-2 pages
adssdadas


\newpage
\section*{Main body}
\addcontentsline{toc}{section}{Main body}
5 pages

\newpage
\section*{Conclusion}
\addcontentsline{toc}{section}{Conclusion}
1 pages
%


\newpage
\renewcommand\refname{Bibliography}
\bibliographystyle{plain}
\bibliography{draft}



\end{document}
